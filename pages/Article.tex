\section{The 32 Articles of the English Catholic Faith}
\fancyhead[RE,LO]{The 32 Articles}
\subsection{On Faith in the Most Holy Trinity}
\begin{enumerate}
	\item Of God the Father
	\begin{enumerate}[i.]
		\item There is one living and true God, the Father Almighty, everlasting, without body, parts, or passions; of infinite power, wisdom, and goodness; the Maker, and Preserver of all things both visible and invisible.
		\item The Father, himself unoriginate and uncaused, is the fountainhead (principium | ἀρχή) of the Trinity. The Father, being eternally Father, thus eternally, wholly, and perfectly begets his Son, who is the Word (Verbum | λόγος).
		\item Likewise, the Word is never without the Spirit. Therefore, God is Trinity: the Father simultaneously generating his Son and eternally, wholly, and perfectly spirating the Holy Ghost.
	\end{enumerate}
	\item Of God the Son
	\begin{enumerate}[i.]
		\item The Son, the Word of the Father, is begotten from everlasting of the Father, the very and eternal God, and of one substance with the Father.
		\item The Most Holy Trinity is the unique object of worship of the Catholic Church, because out of his infinite kindness and mercy, the Word made himself known to man by uniting human nature to his divine person. He is rightly said to be---as St. Justinian sets forth ---a single divine person who has a divine nature wholly from His Father to which is united, by composition and without division, an immaculate human nature wholly from Mary, his Mother.\fn{``For all the holy fathers in harmony teach us that nature or essence and form is one thing and hypostasis or person another, and that nature or essence and form indicate the universal, while hypostasis or person indicate the individual.''} Through the composite Christ, man is made able to truly partake of the divine nature.
		\item The Son, which is the Word of the Father, begotten from everlasting of the Father, the very and eternal God, and of one substance with the Father, took Man's nature in the womb of the blessed Virgin, of her substance: so that two whole and perfect Natures, that is to say, the Godhead and Manhood, were joined together in one Person, never to be divided, whereof is one Christ, very God, and very Man; who truly suffered, was crucified, dead, and buried, to reconcile his Father to us, and to be a sacrifice, not only for original guilt, but also for actual sins of men.
	\end{enumerate}
	\item Of God the Holy Ghost
	\begin{enumerate}[i.]
		\item The Holy Ghost, as St. John Damascene teaches, ``proceeds from the Father and rests in the Son.''\sn{On the Orthodox Faith I:8} As Sts. John Damascene and Photios teach, since the procession of the Holy Ghost is perfect, whole, and singular, the Father alone is the hypostatic origin (αἰτῐ́ᾱ) of the Holy Ghost.
		\item Likewise, as Sts. Augustine and John Damascene and Athanasius and Basil and Photios teach, the Son and Holy Ghost exist in eternal relationship with each other due to (a) their consubstantiality by their shared origin in the Father, (b) the mutual indwelling (circumincessio | περιχώρησις) of each person in the others, and (c) the unique rest the Holy Ghost has eternally in the Son.
		\item The Holy Ghost is the Gift of the Father for the Son. Therefore, he has the Father as his sole eternal efficient cause and the Son as his sole eternal final cause.
			\begin{enumerate}[a.]
				\item As St. Augustine teaches, it can be rightly said that the Holy Ghost proceeds (procedit) from both the Father and the Son. This understanding of procession indicates not origin but advancement of the Holy Ghost in the life of the Holy Trinity, so that---according to this Latin sense---the Holy Ghost proceeds from the Father as principle (principium, ᾰ̓ρχή),\sn{On the Trinity XV:26:47} from the Son as manifesting the communion between the Father and the Son, and from himself as the one who advances and exists in the eternal love of the Three Divine Persons.
				\item While the Latin sense of procession can be understood in an orthodox manner, as the West did of old, it is neither canonical nor wise for such theological speculation to be inserted into the Creed without the consent of an ecumenical council.
			\end{enumerate}
		\item Therefore, the Holy Ghost is of the same substance, majesty, glory, and operation (ἐνέργειᾰ) with the Father and the Son.
	\end{enumerate}
	\item Of the Unknowability of the Divine Nature
	\begin{enumerate}[i.]
		\item The nature of God is wholly and entirely incomprehensible. As St. Paul teaches, God ``alone has immortality, who dwells in unapproachable light, whom no one has ever seen or can see.''\sn{1 Tim. 6:16} Because of this---as St. John Damascene teaches---we can know the qualities surrounding the divine nature (τᾰ́ περῐ́ τήν φῠ́σῐν) but not the nature itself. Instead, it is only possible to know what the divine nature is not.
		\item Man comes to know God through seeing the co-equal and identical divine work of the three persons. As St. Gregory of Nyssa teaches, since the divine nature ``is exalted above the understanding of the questioners . . . it is absolutely necessary for us to be guided to the investigation of the Divine nature by its operations.''\sn{On the Holy Trinity 6}
		\item As the ecumenical councils---especially Chalcedon and Constantinople II---and St. Justinian teach, nature is not alone but hypostatised. Therefore, three hypostases subsist in the divine nature. Since the Father is not the Son and neither the Holy Ghost, an absolute lack of distinction between nature and person cannot be confessed but rather a true divine simplicity which confesses distinction between the persons themselves and between the persons and their unique common nature.
	\end{enumerate}
\end{enumerate}
\subsection{God's Revelation to Man}
\begin{enumerate}
\setcounter{enumi}{4}
	\item Of God's Knowability
	\begin{enumerate}[i.]
		\item Just as nature never exists alone but is hypostatised, likewise nature is never sterile and lifeless but energetic and operational. Therefore, in God, the incomprehensible divine nature is tri-hypostatised and eternally operating.
		\item This operation of God, just as the nature and hypostases, is truly divine and uncreated. By the Most Holy Trinity's divine operation, he gives himself freely to the regenerate out of divine clemency.
		\item Since the divine nature is incomprehensible, man is united to God through the Holy Ghost working in man so that man might become by grace what God is by nature.
		\item These operations of God---being divine and uncreated---neither divide the Holy Trinity into parts nor compromise the divine simplicity.
	\end{enumerate}
	\item Of the Purpose of the Incarnation
	\begin{enumerate}[i.]
		\item As St. Athanasius teaches, the Word becomes man for the salvation of mankind from sin and death. By the union of mankind and divinity in the person of Jesus Christ. The last Adam recapitulates in Himself wholly, perfectly, and charitably the failed life of the first Adam.
		\item At the Cross, Jesus Christ offers a full, perfect, and sufficient sacrifice as a propitiation to the Most Holy Trinity for the sins of the world. Full, perfect, and sufficient, because the infinite riches of God are offered, so that---unlike the sacrifices of the old covenant---there is nothing lacking in its efficacy, scope, or ability to achieve its purpose (the forgiveness of sins). Sacrifice, because the Victim is offered and suffers death by man unto God. Propitiation, because the sacrifice of the man is of infinite worth and provides for all that is lacking for the reconciliation between guilty man and the infinitely holy God. So that in this sweet exchange, as the Epistle to Diognetus terms it, ``the wickedness of many should be hid in a single righteous One, and that the righteousness of One should justify many transgressors!''
		\item Jesus Christ, receiving the Holy Ghost from his Father, likewise gives freely the Holy Ghost to his Body---the Church.
	\end{enumerate}
	\item Of the going down of Christ into Hell
	\begin{enumerate}[i.]
		\item As Christ died for us, and was buried, so also is it to be believed, that he went down into Hell to preach the Gospel to the souls there.
	\end{enumerate}
	\item Of the Resurrection of Christ
	\begin{enumerate}[i.]
		\item Christ did truly rise again from death, and took again his body, with flesh, bones, and all things appertaining to the perfection of Man's nature; wherewith he ascended into Heaven, and there sits, until he return to judge all Men at the last day.
		\item Human nature is so perfectly united to the Divine Word that all men, by the power of God, will be resurrected on the last day. As the Athanasian Creed confesses, ``At whose coming all men shall rise again with their bodies, and shall give account for their own works. And they that have done good shall go into life everlasting, and they that have done evil into everlasting fire.''
	\end{enumerate}
\item Of the Sufficiency of the Holy Scriptures
	\begin{enumerate}[i.]
		\item As St. Athanasius teaches, ``the sacred and inspired Scriptures are sufficient to declare the truth,''\sn{Against the Heathen 1:1} and contains all things necessary to salvation: so that whatsoever is not read therein, nor may be proved thereby, is not to be required of any man, that it should be believed as an article of the Faith, or be thought requisite or necessary to salvation.
		\item The traditions of the Church continue according to the guidance of the Spirit of Truth. Therefore, the authentic traditions of the Catholic Church cannot be rejected without impiety and error.
		\item As St. Augustine teaches, ``holy Scripture sets a rule to our teaching, that we dare not be wise more than it behooves to be wise . . . Be it not therefore for me to teach you any other thing, save to expound to you the words of the Teacher, and to treat of them as the Lord shall have given to me.''\sn{Of the Good of Widowhood 2}
		\item In the name of the Holy Scripture we do understand those canonical Books of the Old and New Testament, of whose authority was never any doubt in the Church. The Church rightly heeds the command of St. John Damascene to ``observe, further, that there are two and twenty books of the Old Testament, one for each letter of the Hebrew tongue,''\sn{On the Orthodox Faith IV:17}\fn{``For there are twenty-two letters of which five are double, and so they come to be twenty-seven. For the letters Caph, Mem, Nun, Pe , Sade are double. And thus the number of the books in this way is twenty-two, but is found to be twenty-seven because of the double character of five.''} as follows:
	\begin{multicols}{3}
	\begin{enumerate}[1.]
		\item Genesis
		\item Exodus
		\item Leviticus
		\item Numbers
		\item Deuteronomy
		\item Joshua
		\item Judges \& Ruth
		\item 1 \& 2 Samuel
		\item 1 \& 2 Kings
		\item 1 \& 2 Chronicles
		\item Job
		\item Psalms
		\item Proverbs
		\item Ecclesiastes
		\item Songs of Solomon
		\item The Twelve Prophets
		\item Isaiah
		\item Jeremiah
		\item Ezekiel
		\item Daniel
		\item 1 \& 2 Esdras
		\item Esther
	\end{enumerate}
	\end{multicols}
	\begin{multicols}{3}[All the Books of the New Testament, as they are commonly received, we do receive, and account them Canonical, being the following:]
	\begin{enumerate}[1.]
		\item Matthew
		\item Mark
		\item Luke
		\item John
		\item Acts of Apostles
		\item Romans
		\item 1 Corinthians
		\item 2 Corinthians
		\item Galatians
		\item Ephesians
		\item Philippians
		\item Colossians
		\item 1 Thessalonians
		\item 2 Thessalonians
		\item 1 Timothy
		\item 2 Timothy
		\item Titus
		\item Philemon
		\item Hebrews
		\item James
		\item 1 Peter
		\item 2 Peter
		\item 1 John
		\item 2 John
		\item 3 John
		\item Jude
		\item Revelation
	\end{enumerate}
	\end{multicols}
	\begin{multicols}{3}[And the other Books (as St. Jerome teaches) the Church reads for example of life and instruction of manners; but yet it does not apply them to establish any doctrine. These books, called ``Apocrypha'' by St. Jerome and ``Panaretus'' by St. John Damascene, ``are virtuous and noble, but are not counted nor were they placed in the ark,''\sn{On the Orthodox Faith IV:17} such are these following:]
		\begin{enumerate}[1.]
			\item 3 Esdras
			\item 4 Esdras
			\item Tobit
			\item Judith
			\item Psalm 151
			\item The rest of Esther
			\item Wisdom of Solomon
			\item Wisdom of Sirach
			\item Baruch the Prophet
			\item Epistle of Jeremiah
			\item Three Children's Song
			\item Story of Susanna
			\item Bel and the Dragon
			\item Prayer of Manasseh	
			\item 1 Maccabees
			\item 2 Maccabees
			\item 3 Maccabees
			\item 4 Maccabees
		\end{enumerate}
		\end{multicols}
		\item However, due to the catholic consent of the Church in her use of these books, they must be considered, even if not inspired, devoid of all impiety or error regarding faith and morals.
		\item The Sacred Scriptures, having God for its author---as a whole and in all of its parts---being perfectly authored by the same Holy Ghost who guides the Catholic Church, and being reposed within the same Church by the same Spirit, must always be interpreted according to the wisdom of the ecumenical councils and the unanimous consent of the Fathers, for the Catholic Church can never come to the knowledge of God except in a catholic mode.\fn{``It behooves those who preside over the churches, every day but especially on Lord's days, to teach all the clergy and people words of piety and of right religion, gathering out of holy Scripture meditations and determinations of the truth, and not going beyond the limits now fixed, nor varying from the tradition of the God-bearing fathers. And if any controversy in regard to Scripture shall have been raised, let them not interpret it otherwise than as the lights and doctors of the church in their writings have expounded it, and in those let them glory rather than in composing things out of their own heads, lest through their lack of skill they may have departed from what was fitting. For through the doctrine of the aforesaid fathers, the people coming to the knowledge of what is good and desirable, as well as what is useless and to be rejected, will remodel their life for the better, and not be led by ignorance, but applying their minds to the doctrine, they will take heed that no evil befall them and work out their salvation in fear of impending punishment.'' -Canon 19 of the Quinisext Council}
	\end{enumerate}
	\item Of the Old Testament.
	\begin{enumerate}[i.]
		\item The Old Testament is not contrary to the New: for both in the Old and New Testament everlasting life is offered to Mankind by Christ, who is the only Mediator between God and Man, being both God and Man. Wherefore they are not to be heard, which feign that the old Fathers did look only for transitory promises. Although the Law given from God by Moses, as touching Ceremonies and Rites, do not bind Christian men, nor the Civil precepts thereof ought of necessity to be received in any commonwealth; yet notwithstanding, no Christian man whatsoever is free from the obedience of the Commandments which are called Moral.
	\end{enumerate}
\end{enumerate}
\subsection{The Catholic Church}
\begin{enumerate}
\setcounter{enumi}{10}
	\item The Institution \& Mission of the Catholic Church
	\begin{enumerate}[i.]
		\item The Catholic Church is ``the Body of Christ,''\sn{1 Cor. 12:27; Eph. 4:12} because she is his bride, for whom he offers himself.\sn{Eph. 5:25; Rev. 19:7} Likewise, she is rightly called ``the People of God.''\sn{Judg. 20:2, Heb. 4:9} The Catholic Church---guided through the ages from the first Adam unto the last Adam, Jesus Christ, and unto the eschaton---is definitively and perfectly founded by Jesus Christ himself as the congregation of his new, final, and everlasting covenant.
		\item The Catholic Church is rightly called a perfect society (\textit{societas perfecta}). That is, she contains within herself all that is necessary for her purpose. Jesus Christ is the only head of the Catholic Church who governs, teaches, and cares for her.
			\begin{enumerate}[a.]
				\item Each particular church (a catholic bishop and his faithful) is fully a catholic church and the cell of the total society or communion.
			\end{enumerate}
		\item The Catholic Church---founded by Jesus Christ---is, as the West prays in its liturgy, ``firmly established upon the Rock of the Apostolic Confession''\sn{Collect for the Vigil of Sts. Peter \& Paul} by the Most Holy Trinity. Therefore, the apostolic faith is the efficient cause, constitutive element, and rule of the Catholic Church.
		\item The Holy Ghost is rightly called the soul, or animating principle, of the Body of Christ---the Church. The Holy Ghost is promised to be with the Catholic Church always, as the ``Spirit of truth'' who will ``guide you into all the truth.''\sn{Jn 16:12}
		\begin{enumerate}[a.]
			\item Since the Body of Christ is eternal, against which ``the gates of hell shall not prevail,''\sn{Mt 16:18} there can never be a time when the catholic faith is absent from, or abandoned by, the Catholic Church. Necessarily, then, any doctrine which is taught by the Catholic Church and believed, as St. Vincent of Lérins teaches, ``everywhere, always, by all,''\sn{Comm.2:6} cannot fail to be truly the catholic faith.
			\item Therefore, such teaching and reception by the Catholic Church, according to the divine promise in the Scriptures, is infallible.
		\end{enumerate}
		\item The entire work of the Catholic Church, including its liturgies and theology and almsgiving, is in fulfilment of its purpose: the salvation of souls. All its work and activity ultimately terminate in that end.
	\end{enumerate}
	\item Membership in the Catholic Church
	\begin{enumerate}[i.]
		\item The Catholic Church is ``the household of God, built on the foundation of the apostles and prophets, Christ Jesus himself being the cornerstone.''\sn{Eph. 2:19-20}
		\item The Church is rightly called a communion, that is, a joint partnership in the orthodox faith.
		\item A member is incorporated into the Church by Word \& Sacrament. As St. Basil teaches, ``first comes the confession, introducing us to salvation, and baptism follows, setting the seal upon our assent.''\sn{de Spiritu Sancto 12:28}
		\item A member cannot truly be incorporated into the Catholic Church without that catholic faith which is perfected in Baptism.
	\end{enumerate}
	\item The Ministry of the Catholic Church
	\begin{enumerate}[i.]
		\item Jesus Christ is rightly called the Apostle of the Father, for he is sent by the Father with his full authority. Likewise, he is rightly called the King of the Kingdom of God, for he is the natural heir of the kingdom according to his humanity and its rightful head according to his divinity. Furthermore, he is rightly called the ``the Shepherd and Bishop of your souls,''\sn{1 Pet 2:25} for he governs, teaches, and cares for his flock.
		\item Just as the Father sent Jesus Christ with his full authority, so Jesus Christ sent the apostles with his full authority for the governing, teaching, and caring for the particular churches throughout the world.\sn{Jn 20:20-1}
		\item The ministry of bishops (ἐπῐ́σκοποι), in true apostolic succession, continues the shepherding and governing of the churches of God.
		\begin{enumerate}[a.]
			\item As the earliest fathers---such as Sts. Clement and Ignatius---teach, the ministry of the new covenant is the fulfilment of the ministry of the old covenant.
			\item The full episcopal ministry is not contained merely in the Sacrament of Holy Orders. Rather, such sacrament, to be rightly administered, must be accompanied by jurisdiction over a particular church.
			\item The ministry of priests (πρεσβῠ́τεροι) is delegated by the bishops and serves to assist the bishops in their ministry, as delegated.
			\item The ministry of deacons (δῐᾱ́κονοι) is fundamentally a ministry of service at the direction of the bishop.
		\end{enumerate}
		\item According to the infinite wisdom of the Most Holy Trinity in his ordering of human sexuality and each sex's particular and unique and non-interchangeable role and purpose, only men---created by God and born as male---can be ordained into the major orders of bishop, priest, and deacon.
		\item For the good order of the Church, other minor orders, being non-sacramental, are---at different times---created and dissolved.
		\item While the Catholic Church is fully found in each true particular church, the canonical boundaries do not exhaust the Catholic Church.
	\end{enumerate}
	\item The Catholic Church rightly exercises her teaching authority through her bishops and the churches committed to their charge.
			\begin{enumerate}[a.]
				\item The bishops are canonically organised in a hierarchy among each other---in analogy of each particular church---seen most excellently in the Pentarchy of old, when elder Rome, having been orthodox, was truly the archbishop of archbishops.
			\end{enumerate}
		\item Since apostolic succession, in its totality, requires both the sacramental character and true jurisdiction; those who have received the Sacrament of Holy Orders but lack either the orthodox faith or jurisdiction do not have magisterial authority.
		\begin{enumerate}[a.]
			\item These commonly called \textit{episcopi vagantes} do not constitute true heads of particular churches and do not properly form true particular churches of the Catholic Church.
			\item Therefore, the universal teaching and reception of catholic \& orthodox doctrine (to which is ascribed infallibility) is not predicated of either these \textit{episcopi vagantes} or their congregations or any cleric who holds not the orthodox \& catholic faith joined with true \& ordinary jurisdiction.
		\end{enumerate}
	\item On Apostolic Succesion
		\begin{enumerate}[i.]
				\item Just as the old covenant had the high priest, the priests, and the Levites; likewise, in the new covenant is given a ministry of bishops, priests, and deacons.
				\item Jesus Christ is the true sufficient High Priest of the new covenant. He is the primordial high priest, bishop, and shepherd.
				\item Jesus Christ, as high priest, offered the full, perfect, and sufficient sacrifice of atonement in fulfilment of the old covenant shadow of the Feast of the Atonement.
				\item In the new covenant, there continues a need for sanctification (based on that sacrifice and its fruit) and the good order of the Church. Therefore, those sacerdotal and pastoral roles---with the appropriate authority---is held by Christ and delegated to others as his representatives (ᾰ̓πόστολοι).
				\item Therefore, the new covenant ministry can never be considered in addition to Jesus Christ or supplement to Jesus Christ but as the sufficient work of Jesus Christ through the instrument of his ministers.
		\end{enumerate}
	\item Of the Ecumenical Councils
	\begin{enumerate}[i.]
		\item The chief expression of the magisterial authority of the bishops is expressed in the ecumenical councils.
		\item When an ecumenical council teaches the orthodox faith and is received by the Catholic Church, it is rightly considered infallible and authentically ecumenical.
		\item The doctrines and canons of the ecumenical councils are binding upon the Catholic Church.
		\item There have been seven ancient and liturgically commemorated ecumenical councils.
		\begin{enumerate}[1.]
			\item Nicaea I
			\item Constantinople I
			\item Ephesus
			\item Chalcedon
			\item Constantinople II (including the Quinisext Council)
			\item Constantinople III
			\item Nicaea II
		\end{enumerate}
		\item Likewise, there have been others councils which have rightly taught the catholic faith and have been received by the Catholic Church:
		\begin{enumerate}[1.]
		\setcounter{enumiii}{7}
			\item Constantinople IV (879-880)
			\item Constantinople V (1351)
		\end{enumerate}
		\end{enumerate}
	\item Of the Creeds
	\begin{enumerate}[i.]
		\item The Three Creeds, Nicene Creed, Athanasius' Creed, and that which is commonly called the Apostles' Creed ought thoroughly to be received and believed: for they may be proved by most certain warrants of Holy Scripture and through the catholic consent of the Church.
	\end{enumerate}
	\end{enumerate}
\subsection{The Salvation of Man}
\begin{enumerate}
\setcounter{enumi}{18}
	\item Of Original Sin.
	\begin{enumerate}[i.]
		\item As all Mankind, during the state of Innocence, was in Adam; so in him all Men, falling from what he fell, remained in a State of Sin. Wherefore Mankind is become, not only subject unto Sin, but also, on Account of Sin, unto Punishment; which, according to the Sentence pronounced of GOD, was (Gen. 2:17), In the Day that thou eatest of the Tree, thou shalt surely die. And to this the Apostle alludes (Rom. 5:12), Wherefore as by one Man Sin entered into the World, and Death by Sin, and so Death passed upon all Men, for that all have sinned. So that we are conceived in our Mother’s Womb, and born in this Sin, according to the holy Psalmist (Psal. 51:7), Behold, I was shapen in Wickedness, and in Sin hath my Mother conceived me. This is called parental, or original Sin, first, because that, before this, Man was free from all Sin; although the Devil was then corrupt, and fallen, by whose Temptation this parental Sin sprang up in Man; and Adam becoming guilty, we all likewise, who descend from him, become also guilty. Secondly, this is called original Sin, because no Mortal is conceived without this Depravity of Nature.
		\item Original sin standeth not in the following of Adam, (as the Pelagians do vainly talk;) but it is the fault and corruption of the Nature of every man, that naturally is engendered of the offspring of Adam; whereby man is very far gone from original righteousness, and is of his own nature inclined to evil, so that the flesh lusteth always contrary to the Spirit; and therefore in every person born into this world, it deserveth God's wrath and damnation. And this infection of nature doth remain, yea in them that are regenerated; whereby the lust of the flesh, called in Greek, φρονημα σαρκος, (which some do expound the wisdom, some sensuality, some the affection, some the desire, of the flesh), is not subject to the Law of God. And although there is no condemnation for them that believe and are baptized; yet the Apostle doth confess, that concupiscence and lust hath of itself the nature of sin.
	\end{enumerate}
	\item Of Free-Will
	\begin{enumerate}[i.]
		\item The condition of Man after the fall of Adam is such, that he cannot turn and prepare himself, by his own natural strength and good works, to faith; and calling upon God. Wherefore we have no power to do good works pleasant and acceptable to God, without the grace of God by Christ preventing us, that we may have a good will, and working with us, when we have that good will.
	\end{enumerate}
	\item Of the Justification of Man
	\begin{enumerate}[i.]
		\item God, in his mercy, works in man to unite him through the Holy Ghost to Christ in reconciliation with the Father.
		\item This divinisation (θέωσις) begins at justification when man, having true confidence in the divine promises worked in him, is accounted righteous before God, only for the merit of our Lord and Saviour Jesus Christ by Faith, and not for his own works or deservings.
		\item When the Most Holy Trinity declares a man to be righteous, there is both a legal and a true change wrought in the man himself, since the Lord's words can never be without effect.
		\item As attested by the Sacred Scriptures and the Fathers, man is united to Jesus Christ and becomes by grace what he is by nature. However, sin and darkness co-exist with this divinisation as long as faith remains, for---as St. Basil teaches---``what saves us is our faith.''\sn{De Spiritu Sancto 18} And as St. Aelred teaches, ``let faith be for us, then, like the first day on which we believers are separated from unbelievers, as light from darkness,''\sn{Speculum Caritatis I:32:90} even though it is not until the seventh day---the eschaton---that ``all the darkness of error shall be dispersed.''\sn{Speculum Caritatis I:32:94}
		\item Faith (πῐ́στῐς) is not a mere intellectual assent to divinely revealed truths but a confidence and trust in the promises of God which is perfected and quickened by repentance. Mere intellectual assent is not worthy of the name ``faith.''
		\item As St. Basil teaches, divinising faith can exist in man before Baptism through the preaching of the Word, for ``faith is perfected through baptism, baptism is established through faith, and both are completed by the same names.''\sn{De Spiritu Sancto 12}
	\end{enumerate}
	\item Of Sanctification
	\begin{enumerate}[i.]
		\item After justification---or as the fulfilment of justification---the Holy Ghost further divinises the Christian through leading him into greater obedience to, and love of, God. Therefore, faith is necessarily fecund in producing good works, which are the fruits of Faith, and follow after Justification.
		\item Good works, not reconciling man with God as if lively faith was insufficient, heal the effects of sin in broken human nature. Since man is in Christ, and Christ lives in him, good works are pleasing and acceptable to God in Christ, and do spring out necessarily of a true and lively Faith insomuch that by them a lively Faith may be as evidently known as a tree discerned by the fruit.
		\item Therefore, where good works are not seen but, rather, wicked works and neglect of Christ's promises, the Catholic Church rightly judges a wicked man to be faithless and in need of penance.
	\end{enumerate}
	\item Of Glorification
	\begin{enumerate}[i.]
		\item Those elect, chosen ``in him before the foundation of the world,''\sn{Eph 1:4} who are justified and sanctified are, on the last day, reunited with their bodies and glorified.
		\item Such glorification is a full and complete divinisation of the human person.
		\item Before the eschaton, but after death, the full reward and judgement is not given to the faithful but a foretaste in Hades.
		\item Those who die in faith, united to Jesus Christ, do not fear death, for they trust in the promise that ``you will not abandon my soul to Sheol, or let your holy one see corruption.''\sn{Ps 16:10}
	\end{enumerate}
	\item Of Works before Justification
	\begin{enumerate}[i.]
		\item Works done before the grace of Christ, and the Inspiration of his Spirit, are not pleasant to God, forasmuch as they spring not of faith in Jesus Christ; neither do they make men meet to receive grace, or (as the School-authors say) deserve grace of congruity: yea rather, for that they are not done as God hath willed and commanded them to be done, we doubt not but they have the nature of sin.
	\end{enumerate}
	\item Of Christ without Sin
	\begin{enumerate}[i.]
		\item Christ in the truth of our nature was made like unto us in all things, sin only except, from which he was clearly void, both in his flesh, and in his spirit. He came to be the Lamb without spot, who, by sacrifice of himself once made, should take away the sins of the world; and sin (as Saint John says) was not in him. But all we the rest, although baptised and born again in Christ, yet offend in many things; and if we say we have no sin, we deceive ourselves, and the truth is not in us.
		\item Because Our Lord united an immaculate human nature to himself, it must be confessed that the sole source of his human nature, Mary the ever-Virgin Mother of God, also had an immaculate (Παναγία) human nature.
			\begin{enumerate}
				\item The heresy which supposes that the human nature, destined for the Word, could exist unhypostatised and be cleansed in such a supposed-state must be rejected, for by distorting the doctrine of the Mother of God, it distorts the doctrine of the Incarnate God himself.
				\item Likewise, the heresy which supposes that the human nature, united to Divine Word, was made immaculate by such union must also be rejected, for it would demote Christ the unique High Priest down to the level of every other high priest who ``offered for himself and for the people's sins committed in ignorance.''\sn{Heb 9:7}
			\end{enumerate}
	\end{enumerate}
	\item Of Predestination and Election
	\begin{enumerate}[i.]
		\item Predestination to Life is the everlasting purpose of God, whereby (before the foundations of the world were laid) he hath constantly decreed by his counsel secret to us, to deliver from curse and damnation those whom he hath chosen in Christ out of mankind, and to bring them by Christ to everlasting salvation, as vessels made to honour. Wherefore, they which be endued with so excellent a benefit of God, be called according to God's purpose by his Spirit working in due season: they through Grace obey the calling: they be justified freely: they be made sons of God by adoption: they be made like the image of his only-begotten Son Jesus Christ: they walk religiously in good works, and at length, by God's mercy, they attain to everlasting felicity.
		\item As the godly consideration of Predestination, and our Election in Christ, is full of sweet, pleasant, and unspeakable comfort to godly persons, and such as feel in themselves the working of the Spirit of Christ, mortifying the works of the flesh, and their earthly members, and drawing up their mind to high and heavenly things, as well because it doth greatly establish and confirm their faith of eternal Salvation to be enjoyed through Christ as because it doth fervently kindle their love towards God: So, for curious and carnal persons, lacking the Spirit of Christ, to have continually before their eyes the sentence of God's Predestination, is a most dangerous downfall, whereby the Devil doth thrust them either into desperation, or into wretchlessness of most unclean living, no less perilous than desperation.
		\item Furthermore, we must receive God's promises in such wise, as they be generally set forth to us in Holy Scripture: and, in our doings, that Will of God is to be followed, which we have expressly declared unto us in the Word of God.
	\end{enumerate}
	\end{enumerate}
\subsection{Of the Sacraments}
\begin{enumerate}
\setcounter{enumi}{26}
	\item In Our Lord's new covenant, he hath established seven sacraments for our salvation.
	\item Sacraments ordained of Christ be not only badges or tokens of Christian men's profession, but rather they be certain sure witnesses, and effectual signs of grace, and God's good will towards us, by the which he works invisibly in us, and does not only quicken, but also strengthen and confirm our Faith in him.
	\item The Sacraments were not ordained of Christ to be gazed upon, or to be carried about, but that we should duly use them.
	\item It is impious to celebrate or believe in the Sacraments in a manner contrary to the catholic and orthodox witness of the Fathers and the Church.
	\item There are two properly called Sacraments of the Gospel, for they are most illustriously promised by Jesus Christ in the Gospels and are generally necessary for salvation.
	\begin{enumerate}[i.]
		\item Baptism
			\begin{enumerate}[a.]
				\item Baptism is a mystery by which, when the body is washed in water, the soul of the believer is cleansed from sins by the Blood of Christ.
				\item This mystery was ordained by Christ himself, in the command which He gave his disciples.
				\item By means of this visible act he invisibly receives salvation of the soul, according to the promise of Christ.
			\end{enumerate}
		\item Eucharist
			\begin{enumerate}[a.]
				\item The Eucharist, or Communion, is a mystery in which the believer, under the form of bread, receives the Body itself of Christ; and under the form of wine, the Blood itself of Christ; for the remission of sins, and unto eternal life.
				\item Consequently, every true Christian ought to be persuaded that in this sublime mystery he does not receive simple bread and simple wine; but that, under the form of the hallowed bread, he receives the true Body itself of Christ, which was offered as a sacrifice upon the Cross for our salvation, and, like bread, was broken by many sufferings; and that, under the form of the hallowed wine, he receives the true Blood itself of Christ, which was poured forth from his pure side, and became the propitiation for all our sins. For when He gave the bread to his disciples the Lord said: ``This is my Body;'' and giving the wine He said, ``This is my Blood.''
				\item The Cup of the Lord is not to be denied to the Lay-people: for both the parts of the Lord's Sacrament, by Christ's ordinance and commandment, ought to be ministered to all Christian men alike.
			\end{enumerate}
	\end{enumerate}
	\item There are five others properly called Sacraments of the Church, for they are promised by Jesus Christ but revealed in the working of the early church in the Acts of the Apostles and are for the good order of the Catholic Church.
	\begin{enumerate}[i.]
	\setcounter{enumii}{2}
		\item Confirmation
			\begin{enumerate}[a.]
				\item Confirmation is a mystery by which, when the members of the body receive the imposition of hands and are anointed with chrism, there is poured forth upon the baptised person a spiritual unction; that is to say, the gifts of the Holy Ghost.
				\item This mystery is accomplished immediately after Baptism.
				\item By means of this sacred ordinance the Holy Ghost comes upon the baptised person, and seals---that is, confirms---him in the grace which he received at his Baptism; just as He came upon the disciples of Jesus Christ; and as the disciples themselves put their hands after Baptism on the believers, and by this laying on of the hands of the Apostles, those that were baptised received the Holy Ghost.
			\end{enumerate}
		\item Confession
			\begin{enumerate}[a.]
				\item Repentance is a mystery, through which the believer, knowing his sins, and having a firm trust in the mercies of Jesus Christ, receives the forgiveness of sins from God through the priest.
				%\item Not every deadly sin willingly committed after Baptism is sin against the Holy Ghost, and unpardonable. Wherefore the grant of repentance is not to be denied to such as fall into sin after Baptism. After we have received the Holy Ghost, we may depart from grace given, and fall into sin, and by the grace of God we may arise again, and amend our lives. And therefore they are to be condemned, which say, they can no more sin as long as they live here, or deny the place of forgiveness to such as truly repent.
			\end{enumerate}
		\item Unction of the Sick
			\begin{enumerate}[a.]
				\item Unction of the Sick is a mystery, in which the minister of the Church, anointing the sick person with oil, prays God to alleviate his malady and forgive his sins.
			\end{enumerate}
		\item Holy Orders
			\begin{enumerate}[a.]
				\item Priesthood is a mystery in which the Holy Ghost, by the hands of his ministers, ordains them that be rightly chosen to celebrate the mysteries and feed the flock of Christ.
				%\item In the Church, as a well-regulated society, there is a hierarchy; that is, an ecclesiastical government, constituted of the ecclesiastical presidents and rulers. The flock elect these spiritual rulers, and by this means the Lord Himself chooses the fit man.
				\item This ordination is effected by the invocation of the Holy Ghost, and the laying on of hands, at the time of assembly of the Church; which confirms the election.
				\item This ordination, effected by the imposition of hands, has its origin from the Apostles themselves, from whom it has been handed down from one to another, and come to us.
			\end{enumerate}
		\item Matrimony
			\begin{enumerate}[a.]
				\item Holy Matrimony is a mystery in which the minister of the Church crowns two persons to be united, and prays for the divine blessing upon them.
				\item The minister of the church joins them, and with all the Church beseeches God to grant them love, peace, and the blessing of children. And thus, by means of this ceremony, the bond becomes closer, as being confirmed before the altar of the Lord. This yoke can only be entered into between one man and one woman; for the Christian law does not by any means admit of polygamy.
				\item The intention and end of wedlock is the preservation of the human race. The man, as head of this relationship, should heartily love his wife, regarding her as his most faithful helpmate in the care and nourishment of his children; and not treat her harshly or cruelly, but correct her weakness with prudent forbearance. The duty of a woman is, to love and honour her own husband; that is, to accommodate her manners to his pleasure, and to submit even to ill-treatment with meekness of spirit. And they ought both to keep their marriage-bed blameless, and live together until death separates them.
				\item It is sinful, and truly impossible, to separate the purpose of procreation in the marital act from the purpose of mutual care and unity.
			\end{enumerate}
	\end{enumerate}
\end{enumerate}